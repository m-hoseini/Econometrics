\documentclass{beamer}
\usetheme{CambridgeUS} % replace it with Boadilla if you want no section bar
%\usecolortheme{crane} % other ones: dove, dolphin, rose, seahorse, orchid, crane, seagull, lily, wolverine
%\usefonttheme{serif} 
\usefonttheme[onlymath]{serif} % uncomment if you want it just for math

\setbeamertemplate{navigation symbols}{}  % comment to have nagivation

\usepackage[compress,comma,authoryear]{natbib}
\usepackage{tikz}
\usetikzlibrary{mindmap,trees}
\usepackage{amsmath,mathtools}
\usepackage{amsthm}
\usepackage{booktabs}
\usepackage{graphicx,epstopdf}
\usepackage{hyperref}


\definecolor{blue}{RGB}{0,114,178}
\definecolor{orange}{RGB}{213,94,0}
\definecolor{red}{RGB}{190,0,0}
\definecolor{yellow}{RGB}{240,228,66}
\definecolor{green}{RGB}{0,158,115}
\definecolor{Lblue}{RGB}{0,197,155}
\definecolor{Dblue}{RGB}{0,76,119}
\definecolor{Lgreen}{RGB}{180,255,230}

\hypersetup{
	colorlinks=false,
	linkbordercolor = {white},
	linkcolor = {blue}
}
\definecolor{MyBackground}{RGB}{245,245,245}

\setbeamercolor{frametitle}{fg=blue}
\setbeamercolor{title}{fg=blue}
\setbeamertemplate{footline}[frame number]
\setbeamertemplate{navigation symbols}{} 
\setbeamertemplate{itemize item}[circle]%{$\bigstar$}
\setbeamertemplate{itemize subitem}{$\bigstar$}
\setbeamercolor{itemize item}{fg=blue}
\setbeamercolor{itemize subitem}{fg=blue}
\setbeamercolor{enumerate item}{fg=blue}
\setbeamercolor{enumerate subitem}{fg=blue}
\setbeamercolor{button}{bg=MyBackground,fg=blue}
\setbeamercolor*{palette primary}{use=structure,fg=blue,bg=white}
\setbeamercolor*{palette secondary}{use=structure,fg=white,bg=Dblue}
\setbeamercolor*{palette tertiary}{use=structure,fg=white,bg=blue}
\setbeamercolor*{palette quaternary}{fg=white,bg=black}
\setbeamercolor*{palettes quaternary}{fg=white,bg=Lgreen}
%\setbeamercolor{titlelike}{parent=structure,bg=Lgreen}
%\setbeamercolor{title in head/foot}{bg=Lgreen,fg=orange}

\setbeamertemplate{enumerate item}{%
	\usebeamercolor[bg]{item projected}%
	\raisebox{1.5pt}{\colorbox{blue}{\color{fg}\footnotesize\insertenumlabel}}%
}





\begin{document}
	\title[Econometrics 2]{Econometrics 2 (M.Sc.)}
	\subtitle{Panel Data}
	\author[Mohammad Hoseini]{Mohammad Hoseini}
	
	%\institute[IMPS]{Institute for Management and Planning Studies (IMPS)}
	
	\date[Spring 2024]{Spring 2024 \\
	\vspace{10pt} @metrics2
}
	
\begin{frame}[plain]
	\titlepage
\end{frame}

\section{Panel Data}
\subsection{Introduction}

\begin{frame}{Outline}

\begin{enumerate}
	\item Introduction
	\item Random effect models
	\item Fixed effect models
	\item Practical issues and some applications.
\end{enumerate}\bigskip


\end{frame}

\begin{frame}{Panel data}

Panal data include observations from 2 dimensions (or more):
\begin{itemize}
\item $i$: cross-sectional units like individuals, firms, countries, etc.
\item $t$: time periods like years, quarters, months, etc.
\end{itemize}\medskip

Note that repeated cross-section is not panel data.\bigskip

Micro vs. macro panels
\begin{itemize}
\item In micro panels $i>>t$ (e.g. lots of firms over few time periods).
\item In macro panels $i<t$ like (e.g. few countries over 40 years).
\end{itemize}
In this course we cover methods useful for micro panels. Macro panel methods are closer to the time series econometrics. 
\end{frame}

\begin{frame}{Balanced and unbalanced panels}
Panel data can be balanced or unbalanced:
\begin{itemize}
\item Balanced: all units have data for all available time periods.
\item Unbalanced: units are observed for different amounts of time.
\end{itemize}\medskip

If the panel is unbalanced for reasons that are not entirely random (e.g. because firms with relatively low levels of productivity have relatively high exit rates), then we may
need to take this into account when estimating the model.\bigskip

This can be done by means of a sample selection model that we do not cover in this course.

\end{frame}

\begin{frame}{What can panel data solve?}
Panel data can solve selection-on-unobservables and omitted variable problem when the unobservable factor is {\color{red}\textbf{time-invariant}}.\bigskip

With Panel data we can estimate dynamic equations (e.g. specifications with lagged dependent
variables on the right-hand side).\bigskip

Panel data methods are often used in combination with other techniques such as IV or Differences-in-Differences.
\end{frame}


\begin{frame}{Omitted variable problem and panel data}
Suppose the model is  $\ y_{it} = \beta x_{it} + \eta_{it}$
where $\eta_{it}= a_i+u_{it}$\medskip

Here, $a_i$ is unobserved and our goal is to estimate $\beta$.\medskip

Note that $a_i$ is constant over time for each unit and hence it has no $t$ subscript.\medskip\pause

If we estimate the model with OLS, $a_i$ will be hidden in $\eta_i$.\medskip

The consequence of the unobservable factor $a_i$ can be discussed in two cases:
\begin{itemize}
\item $a_i$ and $x_i$ are uncorrelated \onslide<3->{ $\rightarrow$ no OVB: Random Effect models.}
\item $a_i$ and $x_i$ are correlated \onslide<3->{ $\rightarrow$ OVB: Fixed Effect models.}
\end{itemize} \medskip
\onslide<3->{In RE model, we assume $a_i \sim(a,\sigma^2_a)$ is a random variable. \\
	In FE model, we assume $a_i$ is fixed for each individual $i$.}
%when we have alpha the mean a_i is zero.
\end{frame}

\subsection{Random effect models}
\begin{frame}{Case 1: Corr($a_i$,$x_i)=0$}
If $a_i$ is uncorrelated with $x_{it}$, we don't have omitted variable problem and $a_i$ is just another unobserved factor making up the residual.\bigskip

But, notice that here we have $E[\eta_{it}\eta_{it-s}]\neq 0$. This means we have serial correlation in error terms.
\[E[\eta_{it}\eta_{it-s}]=E[(a_i+u_{it})(a_i+u_{it-s})]=E[a_i^2]=\sigma^2_a \]
\[E[\eta_{it} \eta_{it}']=\Omega=\footnotesize \left[\begin{matrix}
\sigma^2_{a}+\sigma^2_u & \dots & \sigma^2_{a}&\dots&0& \dots & 0\\
\vdots & \ddots & \vdots&\vdots & \vdots & \ddots&\vdots\\
\sigma^2_{a} & \dots & \sigma^2_{a}+\sigma^2_u&\dots&0&\dots&0\\
\vdots&\vdots&\vdots&\vdots&\vdots&\ddots&\vdots\\
0&\dots&0& \dots & \sigma^2_{a}+\sigma^2_u & \dots & \sigma^2_{a}\\
\vdots & \ddots & \vdots&\vdots & \vdots & \ddots&\vdots\\
0&\dots&0&\dots&\sigma^2_{a} & \dots & \sigma^2_{a}+\sigma^2_u\\
\end{matrix} \right] \]
\end{frame}

\begin{frame}{Pooled OLS and Random effect model}

\textbf{Pooled OLS} (\textbf{POLS}) estimator: Simply regress $y$ on $x$ irrespective of panel dimensions. To remove serial correlation, cluster error term at units.

\texttt{. tsset i t \\ . reg y x , cluster(i)}\bigskip

Using POLS with clustering solves for error term correlation, but it doesn't change the coefficient from simple POLS (recall SE issues).\medskip

\textbf{Random Effect (RE)} estimator: GLS estimation with above covariance matrix. It is more efficient than POLS. \medskip

{Feasible GLS (FGLS)}: Since we don't know $\Omega$, in practice FGLS estimates the $\hat{\Omega}$ at step 1 and uses it to find $\beta$ at step 2.

\texttt{. xtreg y x, i(i)}
\end{frame}



\subsection{Fixed effect models}
\begin{frame}{Case 2: Corr($a_i$,$x_i)\neq 0$}
If $a_i$ is correlated with $x_{it}$, we have omitted variable problem.\bigskip

What could we do about the omitted variable problem in cross-sectional data?\pause
\begin{itemize}
\item Find a proxy variable
\item Find a valid IV that is correlated with the elements of $X_{it}$ and uncorrelated with $a_i$.
\item Do a randomized experiment.
\end{itemize}\medskip

Panel data gives additional possibilities to deal with the omitted variables problem.
\begin{itemize}
\item We can eliminate $a_i$ using Fixed Effect models.
%\item Under relatively strong assumptions we can use Random Effect models. 
\end{itemize}

\end{frame}

\begin{frame}{Fixed Effect (FE) model}
\vspace{-10pt}
\[y_{it} = \beta x_{it}+(a_i+u_{it}), \ \ t = 1,2,\dots,T \ , \  i = 1,2,\dots, N \]
Assumptions about unobserved terms:
\begin{itemize}
\item strict exogeneity: $E(x_{it}u_{is}) = 0$ for $s = 1,2,\dots,T$ 
\end{itemize}\medskip

Strict exogeneity rules out feedback from past $u_{is}$ shocks to current $x_{it}$.\medskip

One implication of this is that FE will not yield consistent estimates if $x_{it}$ contains lagged dependent variables ($y_{it-1}, y_{it-2} , \dots$).\bigskip

If the assumption of strict exogeneity does not hold, use lag of variables as instruments to get consistent estimates.  Arellano-Bond estimator does this but we do not cover it.
\end{frame}

\begin{frame}{Estimating Fixed Effect models}
In FE models there are 3 ways to eliminate $a_i$ that causes the error term to be uncorrelated with the regressors:
\begin{itemize}
\item Directly estimate FE model with dummy variables.
\item Within-transformation (within estimator).
\item First differencing estimator
\end{itemize}
\end{frame}

\begin{frame}{Dummy Variable Estimator}
The direct way of estimating FE models is to estimate the $a_i$ using a set of dummies for all $i$ in the sample.\bigskip

This method is especially appealing when $N$ is small or we actually want to estimate the fixed effect of each unit (IQ, ability, productivity, etc.)\medskip

For this, we include $N$ dummies ($D_i$ is one for unit $i$) in the regression and estimate using OLS:
\[y_{it} = \beta x_{it} + \sum_i a_i D_i + u_{it}\]
This is also called the dummy variables estimator.\medskip

The cost is that it is computing power intensive if $N$ is large.

\end{frame}

\begin{frame}{Within estimator}
\[y_{it} = \beta x_{it}+(a_i+u_{it})\]\vspace*{-10pt}
\begin{enumerate}
\item Average equation overtime $t = 1,\dots,T$ $ (\bar{x}_i=\frac{1}{T}\sum_t x_{it}$ and so on):
\[\bar{y}_{i} = \beta \bar{x}_{i}+(a_i+\bar{u}_{i})\]
\item Subtract averaged equation from the main equation to get:
\[(y_{it}-\bar{y}_{i}) =  \beta (x_{it}-\bar{x}_{i})+(u_{it}-\bar{u}_{i})\]	
\item Define $\tilde{y}_{it}=y_{it}-\bar{y}_i$ and $\tilde{x}_{it}=x_{it}-\bar{x}$ and run a regression of $\tilde{y}_{it}$ on $\tilde{x}_{it}$ using OLS: $\tilde{y}_{it}=\beta\tilde{x}_{it}+\tilde{u}_{it}$
\end{enumerate}\medskip
Within estimator is exactly the same estimator as the dummy variable estimator. But without many dummy variables to estimate.
This is very practical when $N$ is large and statistical packages use within estimator by default: 
\texttt{xtreg y x , fe}

\end{frame}


\begin{frame}{First Differencing Estimator}
\[y_{it} =  \beta x_{it}+(a_i+u_{it})\]
\[y_{it-1} =  \beta x_{it-1}+(a_i+u_{it-1})\]
\[y_{it}-y_{it-1} = \beta (x_{it}-x_{it-1})+u_{it}-u_{it-1} \]\[ \Rightarrow \ \Delta y_{it}=\beta \Delta x_{it}+\Delta u_{it} \]
Run a regression of $\Delta y_{it}$ on $\Delta x_{it}$ using OLS.\bigskip

Thus, lagging the model by one period eliminates $a_i$.\bigskip


In differencing we lose the first time period for each cross section and instead of $T$ have $T-1$ time periods for each $i$.
\end{frame}

\begin{frame}{Within estimator vs first differencing estimator}
For $T = 2$ within and first differencing are exactly equivalent.\bigskip

For $T> 2$ they are different
\begin{itemize}
\item Under homoskedasticity i.e. $u_i\sim iid(0,\sigma^2)$, the within estimator is more efficient. 
\item If $u_{it}$ is a random walk ($u_{it}=u_{it-1}+v_{it}$), then $\Delta u_{it}$ is serially uncorrelated and so the FD estimator is more efficient.
\end{itemize}\medskip

If the two estimators are significantly different you should worry about the validity of the strict exogeneity assumption.
\end{frame}

\subsection{Practical issues}
\begin{frame}{Choosing between RE and FE}
The RE assumption that Corr($a_i,x_i)=0$ is very strong and 
POLS and RE estimator are inconsistent if the real model is FE.\medskip

RE model may still be desirable under some circumstances
\begin{itemize}
\item If $N >>T$, in FE you include N dummies and loose N degrees of freedom, but in random effect you loose 2 dof ($a,\sigma^2_a$)
\item In FE, you only estimate within-subject variability ($\tilde{y}_i=\beta\tilde{x}_i+\tilde{u}_i$) and loose any between-subject variabilities ($\bar{y}_i=\beta\bar{x}_i+\bar{u}_i$). If within variability is low (subjects change little overtime), FE may not work well. RE combines within and between variability to estimate.
\item Measurement error is more problematic in FE.
\end{itemize}

To test the hypothesis that $a_i$ is uncorrelated with $x_{it}$, you can use a \textbf{Hausman test}. If the null is rejected you cannot use RE.
\end{frame}

\begin{frame}{Practical tips}
As Corr($a_i,x_i)=0$ is unlikely to hold in many cases we usually prefer FE models.\medskip

In micro panel regression we normally include time dummies. 
\[y_{it} = \alpha+a_i+ b_t + \beta x_{it}+u_{it} \]
Assume we have a binary treatment $x=0,1$. In terms of potential outcome model the above fixed effect regression assumes
\[ E[y^0_{it}|i,t]=a_i+b_t \ \text{  and  } \ E[y^1_{it}|i,t]=a_i+b_t+\beta  \]
When you see an equation with this notation, it means that all individual and times dummies are included in the regression. In fact:
\begin{itemize}
\item $a_i=a_1D_{i=1}+\dots+a_{N-1}D_{i=N-1}$ is a vector of individual dummies
\item $b_t=b_1D_{t=1}+\dots+b_{T-1}D_{t=T-1}$ is a vector of time dummies
\end{itemize} 
\end{frame}

\begin{frame}{Analyzing the magnitude of effect in the regressions}
In many cases, when we run a regression, the variables are in different scales and hard to compare.
To capture the effect size the simplest way is using standard deviation for scaling. \medskip

Consider the simple regression of $n$ explanatory variables:
\[y_i=\beta_0+\beta_1x_{1i}+\dots+\beta_nx_{ni}+\varepsilon_i \ \Rightarrow \ \bar{y}=\beta_1\bar{x}_1+\dots+\beta_n\bar{x}_n \]
\[\frac{y_i-\bar{y}}{\sigma_y}=\beta_1\frac{\sigma_{x_1}}{\sigma_y}\frac{x_{1i}-\bar{x}_1}{\sigma_{x_1}}+\dots+\beta_n\frac{\sigma_{x_n}}{\sigma_y}\frac{x_{1n}-\bar{x}_n}{\sigma_{x_n}} +\frac{\varepsilon_i}{\sigma_y}\]
Define $\tilde{y}_i=(y_i-\bar{y})/\sigma_y$ then \[\tilde{y}_i=\tilde{\beta}_1\tilde{x}_{1i}+\dots+\tilde{\beta}_n\tilde{x}_{ni}+\tilde{\varepsilon}_i \text{ where } \tilde{\beta}_i=\beta_i\frac{\sigma_{x_i}}{\sigma_y} \]
$\tilde{\beta}_i$ is the normalized coefficient: if we increase $x_k$ by one SD ($\sigma_{x_k}$), then $y$ increases by $\tilde{\beta}_k$ SD of $y$ ($\sigma_{y}$).
\end{frame}

\begin{frame}{Effect magnitude in panel regressions}
\[y_{st}=\beta_0+\sum_{i=1}^{S}{a_iS_i}+\sum_{i=1}^{T}{b_iT_i}+\beta_1x_{1st}+\ldots+\beta_nx_{nst}+\varepsilon_{st}\]
\[\text{mean: } \bar{y}=\frac{1}{ST}\sum_{s,t=1}^{S,T}y_{st}, \text{ within  t: } \bar{y}_s=\frac{1}{T}\sum_{t=1}^{T}y_{st},\text{ within s: }\bar{y}_t= \frac{1}{S}\sum_{s=1}^{S}y_{st} \]
\[\bar{y}=\beta_0+\sum_{i=1}^{S}{a_i{\bar{S}}_i}+\sum_{i=1}^{T}{b_i{\bar{T}}_i}+\beta_1{\bar{x}}_1+\ldots+\beta_n{\bar{x}}_n\]
\[ {\bar{y}}_s=\beta_0+\sum_{i=1}^{S}{a_iS_i}+\sum_{i=1}^{T}{b_i{\bar{T}}_i}+\beta_1{\bar{x}}_{1s}+\ldots+\beta_n{\bar{x}}_{ns}+{\bar{\varepsilon}}_s \]
\[{\bar{y}}_t=\beta_0+\sum_{i=1}^{S}{\alpha_i{\bar{S}}_i}+\sum_{i=1}^{T}{\beta_iT_i}+\beta_1{\bar{x}}_{1t}+\ldots+\beta_n{\bar{x}}_{nt}+{\bar{e}}_t\]
\[y_{st}-\ {\bar{y}}_s-{\bar{y}}_t+\bar{y}=\beta_1\left(x_{1st}-{\bar{x}}_{1s}-{\bar{x}}_{1t}+{\bar{x}}_1\right)+\ldots+\beta_n\left(x_{nst}-{\bar{x}}_{ns}-{\bar{x}}_{nt}+{\bar{x}}_n\right)+e_{st}-{\bar{e}}_s-{\bar{e}}_t \]
\end{frame}

\begin{frame}{Effect magnitude in panel regressions}
Define $\tilde{y}_{st}=y_{st}-\ {\bar{y}}_s-{\bar{y}}_t+\bar{y}$ and so on then
\[ {\tilde{y}}_{st}=a_1{\tilde{x}}_{1st}+\ldots+a_n{\tilde{x}}_{nst}+{\tilde{e}}_{st} \]
Therefore, if $\Delta x_k=\Delta \tilde{x}_k=\sigma_{\tilde{x}_k}$  then $\Delta y_k=\Delta \tilde{y}_k=\beta_k\sigma_{\tilde{x}_k}$ is how much variable $x_k$ can explain variation in $y$
\[\frac{\Delta y_k}{\Delta y}=\beta_k\frac{\sigma_{\tilde{x}_k}}{\sigma_y}\]

Note that $\sigma_{{\tilde{x}}_k}$ is de-meaned SD which is different from the simple SD ($\sigma_{x_k}$).
\end{frame}

\begin{frame}{The Effect of Unionization on Wages}
Research question: Do union workers earn higher wages?\bigskip

Selection problem: unionized workers may be different from non-unionized workers (e.g. higher skilled, more experienced).\bigskip

Many of these factors are not observable creating omitted variable bias.\bigskip

We can control for this unobserved factors using FE estimator.\medskip

Freeman (1984) analyzed unionization comparing OLS and FE models. The results suggest a positive selection bias for the union workers.

\end{frame}

\begin{frame}{Freeman (1984)}
\[\text{OLS: }\ Y_{it} = c + \lambda_t + \beta \text{Union}_{it} + \gamma X_{it}' + \eta_{it}\]
\[\text{FE: }\ \ Y_{it} = a_i + \lambda_t + \beta \text{Union}_{it} + \gamma X_{it}' + v_{it}\]
\begin{table}
\centering
\begin{tabular}{lccc}
\hline\hline
Survey&years& OLS& Fixed Effects\\
\hline
CPS &74-75 &0.19& 0.09\\
NLSY& 70-78& 0.28& 0.19\\
PSID& 70-79& 0.23 &0.14\\
QES& 73-77 &0.14& 0.16\\
\hline
\end{tabular}
\end{table}

\end{frame}

\begin{frame}{Measurment error in FE}
OLS results were larger than FE can be due to selection bias.\bigskip

Another plausible explanation is measurement error.\bigskip

Measurement error introduces attenuation bias which biases the coefficient toward zero. 
\begin{itemize}
\item In FE we just use within-subject variation or deviations from the mean.
\item This reduces the signal to noise ratio.
\item Thus measurement error is a more important problem in fixed effect models compared to RE or POLS.
\end{itemize}

In Freeman (1984) union status may be misreported for some individuals in each year. Observed year to year changes in union status for one individual may thus be mostly noise.
\end{frame}

\end{document}
